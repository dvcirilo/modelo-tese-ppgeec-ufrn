%
% ********** Resumo
%

% Usa-se \chapter*, e n�o \chapter, porque este "cap�tulo" n�o deve
% ser numerado.
% Na maioria das vezes, ao inv�s dos comandos LaTeX \chapter e \chapter*,
% deve-se usar as nossas vers�es definidas no arquivo comandos.tex,
% \mychapter e \mychapterast. Isto porque os comandos LaTeX t�m um erro
% que faz com que eles sempre coloquem o n�mero da p�gina no rodap� na
% primeira p�gina do cap�tulo, mesmo que o estilo que estejamos usando
% para numera��o seja outro.
\mychapterast{Resumo}

O resumo deve apresentar ao leitor uma id�ia compacta, mas clara do
trabalho descrito na tese. A defini��o precisa e import�ncia do
problema abordado, os principais objetivos, motiva��es e desafios da
pesquisa s�o bons pontos de partida para o resumo. A estrat�gia ou
metodologia empregada na pesquisa, suas principais contribui��es e os
resultados mais importantes tamb�m devem fazer parte do resumo. Note
que o resumo n�o deve ultrapassar uma p�gina.

\vspace{1.5ex}

{\bf Palavras-chave}: Processamento de texto, \LaTeX,
Prepara��o de Teses, Relat�rios T�cnicos.
%
% ********** Abstract
%

\mychapterast{Abstract}

The abstract must present to the reader a short, but clear idea of the
work being reported in the thesis. The precise definition and
importance of the problem being addressed, the main objectives,
motivations and challenges of the research are a good starting point
for the abstract. The strategy or metodology employed in the research,
its main contributions, and the most important results achieved may be
part of the abstract as well. Notice that the
abstract must not exceed one page.

\vspace{1.5ex}

{\bf Keywords}: Document Processing, \LaTeX, Thesis Preparation,
Technical Reports.
