%
% ********** Página de assinaturas
%

\begin{titlepage}

\begin{center}

\LARGE

\textbf{Sobre a Preparação de Propostas de Tema, Dissertações
e Teses no Programa de Pós-Graduação em Engenharia Elétrica da UFRN}

\vfill

\Large

\textbf{Fulano dos Anzóis Pereira}

\end{center}

\vfill

% O \noindent é para eliminar a tabulação inicial que normalmente é
% colocada na primeira frase dos parágrafos
\noindent
% Descomente a opção que se aplica ao seu caso
% Note que propostas de tema de qualificação nunca têm preâmbulo.
Dissertação de Mestrado
%Tese de Doutorado
aprovada em 31 de fevereiro de 2006 pela banca examinadora composta
pelos seguintes membros:

% Os nomes dos membros da banca examinadora devem ser listados
% na seguinte ordem: orientador, co-orientador (caso haja),
% examinadores externos, examinadores internos. Dentro de uma mesma
% categoria, por ordem alfabética

\begin{center}

\vspace{1.5cm}\rule{0.95\linewidth}{1pt}
\parbox{0.9\linewidth}{%
Prof. Dr. Sicrano Matosinho de Melo (orientador) \dotfill\ DCA/UFRN}

\vspace{1.5cm}\rule{0.95\linewidth}{1pt}
\parbox{0.9\linewidth}{%
Prof. Dr. Beltrano Catandura do Amaral (co-orientador) \dotfill\ DCA/UFRN}

\vspace{1.5cm}\rule{0.95\linewidth}{1pt}
\parbox{0.9\linewidth}{%
Prof. Dr. Clint Stallone da Silva \dotfill\ DCEP/UFFN}

\vspace{1.5cm}\rule{0.95\linewidth}{1pt}
\parbox{0.9\linewidth}{%
Profª Drª Florisbela do Amaral \dotfill\ DCA/UFRN}

\end{center}

\end{titlepage}

%
% ********** Dedicatória
%

% A dedicatória não é obrigatória. Se você tem alguém ou algo que teve
% uma importância fundamental ao longo do seu curso, pode dedicar a ele(a)
% este trabalho. Geralmente não se faz dedicatória a várias pessoas: para
% isso existe a seção de agradecimentos.
% Se não quiser dedicatória, basta excluir o texto entre
% \begin{titlepage} e \end{titlepage}

\begin{titlepage}

\vspace*{\fill}

\hfill
\begin{minipage}{0.5\linewidth}
\begin{flushright}
\large\it
Aos meus filhos, Tico e Teco, pela paciência durante
a realização deste trabalho.
\end{flushright}
\end{minipage}

\vspace*{\fill}

\end{titlepage}

%
% ********** Agradecimentos
%

% Os agradecimentos não são obrigatórios. Se existem pessoas que lhe
% ajudaram ao longo do seu curso, pode incluir um agradecimento.
% Se não quiser agradecimentos, basta excluir o texto após \chapter*{...}

\chapter*{Agradecimentos}
\thispagestyle{empty}

\begin{trivlist}  \itemsep 2ex

\item Ao meu orientador e ao meu co-orientador, professores Sicrano
e Beltrano, sou grato pela orientação.

\item Ao professor Apolônio pela ajuda na revisão deste
modelo de tese.

\item Aos colegas Huguinho, Zezinho e Luizinho pelas sugestões de
modelos de tese.

\item Aos demais colegas de pós-graduação, pelas críticas e sugestões.

\item À minha família pelo apoio durante esta jornada.

\item À CAPES, pelo apoio financeiro.

\end{trivlist}
