%
% ********** Resumo
%

% Usa-se \chapter*, e não \chapter, porque este "capítulo" não deve
% ser numerado.
% Na maioria das vezes, ao invés dos comandos LaTeX \chapter e \chapter*,
% deve-se usar as nossas versões definidas no arquivo comandos.tex,
% \mychapter e \mychapterast. Isto porque os comandos LaTeX têm um erro
% que faz com que eles sempre coloquem o número da página no rodapé na
% primeira página do capítulo, mesmo que o estilo que estejamos usando
% para numeração seja outro.
\mychapterast{Resumo}

O resumo deve apresentar ao leitor uma idéia compacta, mas clara do
trabalho descrito na tese. A definição precisa e importância do
problema abordado, os principais objetivos, motivações e desafios da
pesquisa são bons pontos de partida para o resumo. A estratégia ou
metodologia empregada na pesquisa, suas principais contribuições e os
resultados mais importantes também devem fazer parte do resumo. Note
que o resumo não deve ultrapassar uma página.

\vspace{1.5ex}

{\bf Palavras-chave}: Processamento de texto, \LaTeX,
Preparação de Teses, Relatórios Técnicos.
%
% ********** Abstract
%

\mychapterast{Abstract}

The abstract must present to the reader a short, but clear idea of the
work being reported in the thesis. The precise definition and
importance of the problem being addressed, the main objectives,
motivations and challenges of the research are a good starting point
for the abstract. The strategy or metodology employed in the research,
its main contributions, and the most important results achieved may be
part of the abstract as well. Notice that the
abstract must not exceed one page.

\vspace{1.5ex}

{\bf Keywords}: Document Processing, \LaTeX, Thesis Preparation,
Technical Reports.
