%%
%% Capítulo 1: Modelo de Capítulo
%%

% Está sendo usando o comando \mychapter, que foi definido no arquivo
% comandos.tex. Este comando \mychapter é essencialmente o mesmo que o
% comando \chapter, com a diferença que acrescenta um \thispagestyle{empty}
% após o \chapter. Isto é necessário para corrigir um erro de LaTeX, que
% coloca um número de página no rodapé de todas as páginas iniciais dos
% capítulos, mesmo quando o estilo de numeração escolhido é outro.
\mychapter{Introdução}
\label{Cap:introducao}

Este documento é um modelo para propostas de tema para exames de
qualificação, dissertações de mestrado e teses de doutorado a serem
submetidos ao Programa de Pós-Graduação em Engenharia Elétrica (PPgEE)
da Universidade Federal do Rio Grande do Norte (UFRN).  Procure ler o
texto acompanhando o resultado produzido a partir do código fonte que
o gerou.

Neste capítulo introdutório apresentaremos algumas idéias gerais sobre
como compilar documentos utilizando o processador de textos \LaTeX.

\section{Processando textos com \LaTeX}

O \LaTeX\ não é um editor de texto no sentido convencional. Ele funciona
muito mais como um ``compilador'' de uma linguagem de programação de
textos:%
% Este é um comando para incluir um item no glossário
% Note os % no fim da linha anterior e da próxima
\nomenclature{\LaTeX:}{um poderoso processador de textos}%
\begin{enumerate}
\item Inicialmente você escreve um ``programa'' nesta linguagem de
programação de textos (linguagem \LaTeX), dizendo o conteúdo e a
formatação do seu documento.  Para escrever este ``programa'' pode-se
usar qualquer editor de textos capaz de salvar documentos em formato
texto ASCII puro.
\item O próximo passo é compilar este código fonte, produzindo um arquivo
\texttt{.dvi} que funciona como um ``programa objeto''. Esta compilação
é feita pelo programa \texttt{latex}.
\item Em seguida o arquivo \texttt{.dvi} é convertido para o formato no
qual se deseja produzir o texto: PostScript (usando o programa
\texttt{dvips}), PDF (usando o \texttt{pdflatex}, que já faz esta fase
e a fase anterior) ou visualização na tela (\texttt{xdvi}).
\end{enumerate}
Estas explicações tomaram como base a versão do \LaTeX\ mais comum para
sistemas operacionais Unix. Existem outras implementações tanto para
Unix quanto para Windows, onde os comandos executados são diferentes
mas a idéia geral é sempre a mesma.

Além do \texttt{latex} para compilar o texto, pode ser necessário
executar outros programas, como o \texttt{bibtex} para incluir
automaticamente as referências bibliográficas ou o \texttt{makeindex}
para gerar o glossário. Estes programas devem ser chamados em uma
ordem específica. Para automatizar este processo, é fornecido um
arquivo \texttt{Makefile}, de modo que a compilação completa pode
ser feita utilizando um dos seguintes comandos:%
\nomenclature{\BibTeX:}{uma ferramenta para geração automática de
listas de referências bibliográficas}%
\begin{description}
\item[{\tt make}:] executa a tarefa \emph{par default}, que pode ser
alterada no \texttt{Makefile} para apontar para qualquer uma das seguintes.
\item[{\tt make simples}:] apenas executa o \texttt{latex} uma vez;
\item[{\tt make principal.dvi}:] executa todos os passos e aplicativos
necessários para produzir o arquivo \texttt{.dvi} completo;
\item[{\tt make principal.ps}:] executa todos os passos e aplicativos
necessários para produzir o arquivo PostScript \texttt{principal.ps}
completo;
\item[{\tt make principal.pdf}:] executa todos os passos e aplicativos
necessários para produzir o arquivo PDF \texttt{principal.pdf} completo;
\item[{\tt make clean}:] remove todos os arquivos intermediários gerados
no processo de compilação, inclusive o \texttt{principal.dvi}.
\item[{\tt make realclean}:] além de fazer um \texttt{make clean}, remove
os arquivos \texttt{principal.ps} e \texttt{principal.pdf}.
\end{description}

\section{Organização do texto}

O fecho do capítulo introdutório muitas vezes apresenta uma idéia
global do trabalho, mostrando o que vai ser tratado nos capítulos
subseqüentes\footnote{Contrariando o que muitos acreditam, o trema
ainda não foi abolido do português oficial do Brasil, o que já
aconteceu em Portugal; portanto, deve ser usado em palavras como
seqüência, freqüência e aqüífero.}.

Neste documento, o capítulo~\ref{Cap:estilo} apresenta as diretrizes
gerais sobre a formatação dos textos. O capítulo~\ref{Cap:matematica}
apresenta alguns recursos dp \LaTeX\ para escrever expressões
matemáticas, enquanto o capítulo~\ref{Cap:figuras} trata da inclusão
de tabelas, gráficos e figuras no documento. O
capítulo~\ref{Cap:conclusao}, que faz as vezes de capitulo de
conclusões e perspectivas, mostra alguns exemplos de construção
automática de bibliografias utilizando o aplicativo \BibTeX\ e
menciona fontes adicionais para mais informações.


