%%
%% Capítulo 2: Expressões matemáticas
%%

\mychapter{Expressões matemáticas}
\label{Cap:matematica}

\LaTeX~é insuperável no processamento de expressões matemáticas. Expressões
simples como $2^{n}$ podem ser editadas no próprio texto. Equações
complexas como:
% eqnarray é o comando LaTeX de base para expressões multilinhas com
% alinhamento. O & indica o ponto onde todas as linhas devem ser
% alinhadas. Neste capítulo serão apresentadas outras opções de alinhamento
\begin{eqnarray} \label{eq:PDF:RSR}
  p \left( \gamma \right) & = & \frac{1}{2} \sqrt{\frac{M}{\gamma \bar{\gamma}_{b}}} \frac{1}{ \prod_{i=1}^M {\sqrt{\tilde{\gamma}_i}}}
  \int_0^{\sqrt{M \delta}} \int_0^{\sqrt{M \delta} - r_M } \cdots
  \int_0^{\sqrt{M \delta} - \sum_{i = 3}^M {r_i } } \nonumber \\
  & & p \left( {\frac{\sqrt{M \delta} - \sum_{i = 2}^M {r_i }}{\sqrt{\tilde{\gamma}_1}} ,
  \frac{r_2}{\sqrt{\tilde{\gamma}_2}} , \ldots ,\frac{r_M}{\sqrt{\tilde{\gamma}_M}} } \right)
  \, dr_2 \cdots dr_{M-1} \, dr_M
\end{eqnarray}
% sem linha em branco
ou:
% sem linha em branco
\begin{equation} \label{eq:TrCGI}
  T(r) = \frac{1}{f_m}
  \left( \frac{\pi}{2} \sum_{i=1}^M
  {\tilde{r}_i^2 \dot{\varsigma}_i^2}\right)^{-1/2}
  \frac
  {\begin{array}{ll}
  \int_0^{\rho \sqrt{M}} \int_0^{\rho \sqrt{M} - r_M } \cdots
  \int_0^{\rho \sqrt{M} - \sum_{i = 3}^M {r_i } } \int_0^{\rho \sqrt{M} -
  \sum_{i = 2}^M {r_i } }  \\
  p \left( {\frac{r_1}{\tilde{r}_1} ,
  \frac{r_2}{\tilde{r}_2} , \ldots ,\frac{r_M}{\tilde{r}_M} } \right)
  \, dr_1 \, dr_2 \cdots dr_{M-1} \, dr_M \\ \end{array}}
  {\begin{array}{ll}
  \int_0^{\rho \sqrt{M}} \int_0^{\rho \sqrt{M} - r_M } \cdots
  \int_0^{\rho \sqrt{M} - \sum_{i = 3}^M {r_i } } \\
  p \left( {\frac{\rho \sqrt{M} - \sum_{i = 2}^M {r_i }}{\tilde{r}_1} ,
  \frac{r_2}{\tilde{r}_2} , \ldots ,\frac{r_M}{\tilde{r}_M} } \right)
  \, dr_2 \cdots dr_{M-1} \, dr_M \\ \end{array}}
\end{equation}

% ops, aqui eu deixei uma linha em branco de propósito
são automaticamente numeradas e podem ser referenciadas a partir do
texto. Por exemplo, a equação~\ref{eq:TrCGI} é trivialmente derivada
da equação~\ref{eq:PDF:RSR}.

No parágrafo anterior foi intencionalmente introduzido um erro. Note
que o trecho de frase ``são automaticamente numeradas\dots'' tem uma
indentação, ou seja, um espaço inicial de tabulação, como se fosse a
primeira frase de um novo parágrafo. Na realidade, todo o texto
anterior constitui um único parágrafo, no meio do qual se inserem as
duas equações. Por esta razão, os trechos de frase após as equações
não devem se iniciar com letra maiúscula nem ser indentados. É
importante lembrar que nestas situações em que frases são
interrompidas por equações é obrigatória a inclusão de dois pontos no
fim dos trechos da frase, como em ``\dots Equações complexas como:'' e
em ``ou:''.

O que gerou este erro de indentação? Lembre-se que em \LaTeX\ uma
linha em branco no código fonte indica a separação entre dois
parágrafos. A causa do problema é a linha em branco entre o
\verb|\end{equation}| e o \texttt{são automaticamente
numeradas}\dots~. Como regra geral, enquanto um parágrafo não for
encerrado, não podem ser incluídas linhas em branco, mesmo que no meio
do parágrafo existam equações, figuras, notas de rodapé, etc.

Pequenas expressões matemáticas como $x_0^2$ podem ser inseridas
diretamente no texto, delimitadas por cifrões (\texttt{\$}). Deve-se
evitar este recurso com expressões muito grandes, como
% array é o ambiente para fazer coisas como matrizes. É muito similar
% ao tabular, que será explicado no próximo capítulo, com a diferença
% que os elementos do array são expressões matemáticas.
% Para fazer uma matriz , lembre-se que o array não inclui os
% delimitadores. No caso, nós pusemos o array dentro de delimitadores
% \left[ e right]. Estes delimitadores desenham um colchete do tamanho
% necessário para incluir a coisa delimitada
$\left[\begin{array}{cc} 1 & \frac{2}{x+1} \\ -2 &
1\end{array}\right]^{-1}$, porque o espaçamento entre as linhas fica
prejudicado. Para incluir expressões não numeradas maiores, pode-se
utilizar o par de delimitadores \verb|\[| e \verb|\]|, o que gera
expressões centralizadas na página:
\[
% Nesta expressão nós trocamos o delimitador das matrizes, passando
% a usar parênteses
\left(\begin{array}{cc}
1 & \frac{2}{x+1} \\ -2 & 1
\end{array}\right)^{-1} = \left(\begin{array}{cc}
\frac{x+1}{x+5} & -\frac{2}{x+5} \\ \frac{2(x+1)}{x+5} & \frac{x+1}{x+5}
\end{array}\right) \quad
% quad é uma das formas de incluir espaço em expressões matemáticas,
% pois os espaços são ignorados. quad deve ser usado com moderação,
% pois na maioria dos casos a formatação do LaTeX é a mais adequada.
% Muitos usos de quad podem ser substituídos por ambientes de alinhamento,
% explicados a seguir
\text{se} \quad x \neq -1 \quad \text{e} \quad x \neq -5
\]

Um erro comum em expressões matemáticas é o de digitar os nomes de
funções diretamente, sem utilizar os comandos apropriados. Por
exemplo, a expressão $\sin(\omega t+\phi)$ está correta, enquanto que
a expressão $sin(\omega t+\phi)$ é interpretada pelo \LaTeX como sendo
o produto das variáveis $s$, $i$ e $n$ pela expressão entre
parênteses. Para as funções usuais já existem comandos predefinidos,
como \verb|\sin|. Para suas próprias funções, utilize o comando
\verb|\operatorname{}|, diretamente ou definindo um novo comando:
% newcommand define novos comandos, que podem passar a ser usados da
% mesma forma que os comandos LaTeX de base.
\newcommand{\fat}{\operatorname{fatorial}}
\[
\fat(x) = x \cdot \fat(x-1)
\]

O \LaTeX\ possui uma sintaxe específica para índices (sub-escritos) e
expoentes (super-escritos) posicionados do lado direito do objeto a
que se referem, mas não do lado esquerdo. Para conseguir este efeito,
adicione índices e/ou expoentes a um bloco vazio posicionado antes do
objeto:
% O ambiente xalignat* é normalmente utilizado para escrever expressões
% matemáticas multilinhas com vários pontos de alinhamento. Será detalhado
% na próxima seção. Neste caso, está sendo usado de um modo não
% convencional, pois se trata de uma expressão só com uma linha. O ambiente
% xalignat* está sendo empregado aqui para espaçar igualmente os 6 ítens
% da expressão ao longo da linha de texto.
\begin{xalignat*}{3}
    &w^2 &     &w_i &   {}_z&w \\
{}^*&w   & {}_P&w^Q & {}^A_B&w^C_D
\end{xalignat*}

\section{Equações}
\label{Sec:equacoes}

As equações são delimitadas por \verb|\begin{equation}| e
\verb|\end{equation}|. Devem ser identificadas por um \verb|\label|
para permitir referências futuras:
% Nesta equação são utilizados delimitadores. Não se pode usar um
% delimitador direito sem o esquerdo correspondente, mas os delimitadores
% não precisam ser do mesmo tipo (posso abrir com chave e fechar com
% colchete, por exemplo). Existe o delimitador . que é um delimitador
% vazio. Para colocar uma barra vertical ao lado das expressões, colocamos
% um delimitador . à esquerda e um delimitador | à direita.
\begin{equation}
y = g(x) = g(x_{PO}) + \left.\frac{dg}{dx}\right|_{x=x_{PO}}
\frac{(x-x_{PO})}{1!} + \left.\frac{d^2g}{dx^2}\right|_{x=x_{PO}}
\frac{(x-x_{PO})^2}{2!} + \cdots
\label{Eq:Taylor}
\end{equation}

Devem-se evitar referências com expressões do tipo ``a equação acima''
e ``a próxima equação'', pois modificações no texto podem tornar a
referência inválida. Use sempre referências pelo rótulo
(\texttt{label}), como em ``a equação~\ref{Eq:Taylor}'', mesmo para
equações próximas.

Existem vários ambientes para escrever equações, como o \verb|cases|
para construir expressões condicionais:
\begin{equation}
f(x) = 1+\begin{cases}
0   & \text{se $x=0$}\\
1/x & \text{caso contrário}
\end{cases} + \begin{cases}
x/2             & \text{se $x$ é inteiro e par}\\
\frac{x+1}{2}   & \text{se $x$ é inteiro e ímpar}\\
\frac{x+0.5}{2} & \text{se $x$ não é inteiro}
\end{cases}
\end{equation}

\section{Expressões multilinhas}
\label{Sec:multilinhas}

O pacote \texttt{amstex} define vários ambientes para criar expressões
matemáticas que ocupam mais de uma linha. Existem versões dos
ambientes com e sem inclusão do número na equação, conforme indicado
na tabela~\ref{Tab:multilinhas}\footnotemark.
% A explicação sobre o comando \footnotemark está logo mais à frente
% no texto

\begin{table}[htbp]
\begin{center}
\newlength{\LL}
\settowidth{\LL}{Tipo de alinhamento}
\begin{tabular}{|>{\tt}l|>{\tt}l|l|} \hline
\multicolumn{2}{|c|}{PACOTE} &
\multicolumn{1}{c|}{\multirow{2}{\LL}{Tipo de alinhamento}}
\\ \cline{1-2}
\textrm{Com número} & \textrm{Sem número} &  \\ \hline
gather & gather* & sem alinhamento (só múltiplas linhas) \\
multiline & multline* & quebra de equação em várias linhas \\
align & align* & alinhamento em um único ponto \\
alignat & alignat* & alinhamento em vários pontos, no centro da linha\\
xalignat & xalignat* & vários pontos, ocupando toda a linha (com margens)\\
-- & xxalignat & vários pontos, ocupando toda a linha (sem margens)
\\ \hline
\end{tabular}
\end{center}
\caption{Os ambientes para geração de equações multilinhas}
\label{Tab:multilinhas}
\end{table}

Para os ambientes \texttt{alignat}, \texttt{xalignat} e
\texttt{xxalignat} existe um parâmetro obrigatório que é o número de
elementos em cada linha. Cada elemento tem um ponto de alinhamento com
os outros elementos da mesma coluna em outras linhas.  Cada elemento é
separado do próximo por um \texttt{\&}. Dentro de cada elemento, um
novo \texttt{\&} marca seu ponto de alinhamento, conforme o exemplo da
tabela \ref{Tab:exemplomultilinhas}, onde se vê o código fonte e o
resultado produzido.

% Este table não contém um tabular, mas sim uma frase composta por
% dois minipages. Maiores detalhes sobre tables, tabulars, etc. no
% capítulo seguinte.
\begin{table}[htbp] \begin{center}
\hrule
% O ambiente minipage cria uma micropágina com a dimensão especificada
% e que se comporta exatamente como a página real, porém com dimensões
% reduzidas. Ela pode ser incluída em praticamente todo outro ambiente
% e no meio de frases.
\begin{minipage}[c]{0.3\linewidth}
\tiny
% O ambiente verbatim põe na tela o texto exatamente como foi digitado.
% Muito útil para incluir código fonte, como neste caso.
\begin{verbatim}
\begin{xxalignat}{3}
&aaaa  &  mm&mm  &  cccc& \\
xxxx&  &  ii&ii  &  llll&
\end{xxalignat}
\end{verbatim}
\end{minipage}
% para uma minipage ficar à esquerda e a outra à direita vamos
% incluir entre elas dois espaços horizontais \hfill, que são espaços
% que "esticam" até ocupar a máxima largura possível.
\hfill
$\implica$
\hfill
\begin{minipage}[c]{0.3\linewidth}
\begin{xxalignat}{3}
% A primeira coluna alinha o início de um com o fim do outro
% A segunda coluna alinha os dois pelo meio
% A terceira coluna alinha os dois pelo fim
&aaaa  &  mm&mm  &  cccc& \\
xxxx&  &  ii&ii  &  llll&
\end{xxalignat}
\end{minipage}
\hrule
\caption{Exemplo de equação multilinha com vários pontos de alinhamento}
\label{Tab:exemplomultilinhas}
\end{center}\end{table}

Os ambientes da tabela~\ref{Tab:multilinhas} funcionam como um
ambiente \texttt{equation}: ocupam toda a linha e centralizam a
expressão.  Em algumas situações, entretanto, se deseja incluir um
sub-ambiente multilinhas dentro de uma expressão mais geral. Para
isto, o \texttt{amstex} fornece as opções listadas na
tabela~\ref{Tab:submultilinhas}\addtocounter{footnote}{-1}\footnotemark.
% Para inserir uma nota de rodapé, pode-se utilizar o comando \footnote,
% que já insere a marca e o texto. Em algumas situações, entretanto, é
% preciso utilizar um comando para a marca (\footnotemark) e outro para
% o texto (\footnotetext). Uma das situações onde isto é necessário é
% o caso deste exemplo, onde queremos que duas marcas se refiram ao
% mesmo texto.
% O comando \footnotemark incrementa o contador das marcas de rodapé,
% de modo que o primeiro \footnotemark imprimirá a marca 1, o segundo a
% marca 2, etc. Para contornar este incremento automático, nós
% decrementamos o contador de uma unidade antes da segunda chamada.

% Agora vai o texto da nota de rodapé. Ele é associado à marca do
% último \footnotemark. O comando \footnotetext não incrementa o
% contador de notas de rodapé.
\footnotetext{Nestas tabelas foram utilizadas extensões do ambiente
\texttt{tabular} fornecidas pelo pacote \texttt{tabularx}. Esta
extensões serão explicadas no capítulo \ref{Cap:figuras}}

\begin{table}[htbp]
\begin{center}
\begin{tabular}{|>{\tt}l|l|} \hline
% Estes multicolumn's que abranjem uma única coluna são utilizados para
% mudar o tipo de alinhamento. Por default, a primeira coluna é alinhada
% à esquerda (l). Usando o multicolumn, consegue-se que apenas o elemento
% na linha em questão fique centralizado (c).
\multicolumn{1}{|c|}{PACOTE} &
\multicolumn{1}{c|}{Tipo de alinhamento}
\\ \hline
gathered & sem alinhamento (só múltiplas linhas) \\
aligned & alinhamento em um único ponto \\
alignedat & alinhamento em vários pontos
\\ \hline
\end{tabular}
\end{center}
\caption{Os ambientes para geração de trechos multilinhas em equações}
\label{Tab:submultilinhas}
\end{table}

A equação \ref{Eq:alinhamento} ilustra a utilização de ambientes
\texttt{aligned} inseridos dentro de um ambiente \texttt{alignat}.
Nos ambientes multilinhas numerados cada linha terá seu próprio número
e deverá receber seu próprio rótulo (\texttt{label}), exceto caso se
informe explicitamente que a linha não deverá ser numerada, usando o
comando \verb|\nonumber|.

\begin{alignat}{3}
% Primeira linha do alignat (não numerada)
% Primeiro elemento do alignat (inclui um aligned) - alinha no fim
\left\{\begin{aligned} % Alinha no sinal de =
\dot{\mathbf{x}}(t) &= \mathsf{A}\mathbf{x}(t)+\mathsf{B}\mathbf{u}(t) \\
\mathbf{y} &= \mathsf{C}\mathbf{x}(t)+\mathsf{D}\mathbf{u}(t)
\end{aligned}\right.&
% Note que o aligned é envolvido por dois delimitadores diferentes:
% uma chave (\left\{) no lado esquerdo e um vazio (\right.) no direito
&
% Segundo elemento do alignat (só uma seta) - alinha no início
&\implica
&
% Terceiro elemento do alignat (inclui um aligned) - alinha no início
&\left\{\begin{aligned} % Alinha no sinal de =
s\mathbf{X}(s)-\mathbf{x}(0)&=\mathsf{A}\mathbf{X}(s)+\mathsf{B}\mathbf{U}(s)\\
\mathbf{Y}(s) &= \mathsf{C}\mathbf{X}(s)+\mathsf{D}\mathbf{U}(s)
\end{aligned}\right.\implicafim
\nonumber
\\
% Segunda linha do alignat (numerada)
% Primeiro elemento do alignat (inclui um aligned) - alinha no fim
\left\{\begin{aligned} % Alinha no sinal de =
\mathbf{X}(s) &= \Phi(s)\mathbf{x}(0) + \Phi(s)\mathsf{B}\mathbf{U}(s) \\
\mathbf{Y}(s) &= \mathsf{C}\mathbf{X}(s)+\mathsf{D}\mathbf{U}(s)
\end{aligned}\right.&
&
% Segundo elemento do alignat - alinha no início
&\quad \text{onde} \quad
&
% Terceiro elemento do alignat - alinha no início
&\Phi(s) = (s\mathsf{I}-\mathsf{A})^{-1}
\label{Eq:alinhamento}
\end{alignat}
